\documentclass[unicode,12pt,aspectratio=169,dvipdfmx]{beamer}
\usepackage{bxdpx-beamer}
\usetheme[progressbar=frametitle]{metropolis}
\renewcommand{\kanjifamilydefault}{\gtdefault}
\usepackage{bm}

% タイトル、著者、所属、日付の設定 [65]
\title{スマートホーム異常音検知研究計画} % [1]
\author{竹本志恩} % Placeholder
\date{\today}

% --- セクション開始時に目次を表示 ---
% 「For Better Research Talk」[34, 35]の目次に関するアドバイス(短い発表では不要だが、長い場合は有効)と、
% 「介護におけるHARと連合学習」[65, 66]の具体的な実装を参考に、セクション開始時に目次を表示します。
\AtBeginSection[]{
\begin{frame}[plain]
\frametitle{目次} % フレームタイトルに「目次」と表示
\tableofcontents[currentsection, hideallsubsections] % 現在のセクションを表示し、サブセクションは非表示
\end{frame}
}

\begin{document}

% タイトルスライド [34]
\begin{frame}
\titlepage
\end{frame}

% ==============================================================================
\section{研究背景} % [1]
% ==============================================================================
\begin{frame}{社会背景と研究の方向性} % 聴衆が関心を持てるよう、背景や見通しを含めて語る [23]
    \begin{itemize}
        \item \textbf{これまで:}
        \begin{itemize}
            \item 少子高齢化による介護負担増大に対応するため、\textbf{安価で手軽な見守りシステム}が導入されてきました[1]。
            \item その目的は、迅速な対応と監視負担の軽減です[1]。
        \end{itemize}
        \item \textbf{これから:}
        \begin{itemize}
            \item \textbf{Ambient Assisted Living (AAL)} の流れを汲み、ICT技術で高齢者の\textbf{自立した生活を支援}する方向へ[1]。
            \item これを\textbf{行動認識}によって実現することを目指します[1]。
        \end{itemize}
    \end{itemize}
\end{frame}

\begin{frame}{介護・医療負担軽減への貢献} % 伝えるべき内容を明確に [17, 18]
    \begin{itemize}
        \item 少子高齢化により介護・医療負担が増大している中、在宅での介護・医療の重要性が高まっています[1]。
        \item 高齢者が施設ではなく、慣れ親しんだ場所で自立した生活を送れるよう支援が求められます[1]。
        \item 異常行動から早期に病気などを発見・対応することで、介護・医療負担の軽減に繋げます[1]。
        \item この研究は、そのための基盤技術を提供するものです。
    \end{itemize}
\end{frame}

% ==============================================================================
\section{全体ロードマップ} % [1]
% ==============================================================================
\begin{frame}{研究計画の大まかな予定} % 内容のバランスを確認し、時間配分を意識する [21]
    \begin{itemize}
        \item 研究計画書に基づき、大まかな予定を立案しました[1]。
        \item 今後2週間で詳細を詰めます[1]。
        \item \textbf{7月:基礎準備}[1]
        \begin{itemize}
            \item 背景サーベイ[1]
            \item マルチラベル音響イベント検知モデルの段階的実装[1]
            \item 異常イベントの音響データ収集[1]
        \end{itemize}
        \item \textbf{8月:必要な施策を実行}[2]
        \begin{itemize}
            \item データ不均衡への対応[2]
            \item 簡単なFederated Learning (FL) の適用[2]
            \item Federated Semi-Supervised Learningの調査・検討[2]
            \item シミュレーション方法の調査・検討[2]
        \end{itemize}
    \end{itemize}
\end{frame}

\begin{frame}{ロードマップ(続き)} % 連続する情報を複数のスライドに分割 [27]
    \begin{itemize}
        \item \textbf{9月:異常検知層の追加}[2]
        \begin{itemize}
            \item 方法の検討[2]
            \item 異常検知の実装・実験[2]
        \end{itemize}
        \item \textbf{10月:FLの実装・実験}[2]
        \begin{itemize}
            \item 集約法の工夫検討[2]
        \end{itemize}
        \item \textbf{今後の実験評価計画:}「実験評価計画立案のための計画」が起点となります[2]。
        \item \textbf{主要なタスク:}[2]
        \begin{itemize}
            \item モデル実装
            \item データ収集
            \item データ不均衡対策
            \item 半教師あり連合学習
            \item 異常検知モデルの構築
            \item シミュレーションの検討
        \end{itemize}
    \end{itemize}
\end{frame}

% ==============================================================================
\section{現在の進捗} % [2]
% ==============================================================================
\begin{frame}{モデル実装の進捗} % 具体的な内容を説明する [22]
    \begin{itemize}
        \item 「実験評価計画立案のための計画」が起点となります[2]。
        \item \textbf{最低限のマルチラベル音響検知モデルを実装済み}[2]。
        \begin{itemize}
            \item 最初はCNNの簡単なモデル[2]。
            \item 現在は先輩の実装を参考に、\textbf{Transformerベースのモデル}に移行[2]。
        \end{itemize}
        \item \textbf{マルチラベル実装のPoC (Proof of Concept) 作成済み}[2]。
        \begin{itemize}
            \item CNNによる最低限のマルチラベルモデルを作成[3]。
            \item \textbf{マルチラベルデータ生成:}
            \begin{itemize}
                \item ESC-50データセットから14のカテゴリーを使用[3]。
                \item 音の2つの組み合わせを全パターン作成しCSV化(約12000データ)[3]。
                \item 現在、\textbf{9048データ}を学習に使用(ラベル組が均衡になるようにランダム取得)[3]。
                \item \alert{要修正:} 後でテストデータも作成[3]。
            \end{itemize}
        \end{itemize}
    \end{itemize}
\end{frame}

% \begin{frame}{Transformerモデルの定義} % 図や表の代わりにテキストで構造を説明する [32]
%     \begin{itemize}
%         \item モデルの定義はSlackで説明済み[3]。\alert{要理解:} モデル定義の意味を理解する[3]。
%         \item 大まかなデータフロー(Mermaid図を参考に再構成)[3-6]:
%         \begin{itemize}
%             \item \textbf{Input:} (Batch, Frequencies, Time) の音響データ[3]。
%             \item \textbf{BatchNorm1d:} 周波数次元で正規化[3]。
%             \item \textbf{Permute:} 時間軸と周波数軸を入れ替え(Batch, Time, Frequencies)[3]。
%             \item \textbf{Linear:} 周波数次元を埋め込み次元 (`embed_dim`) に変換[4]。
%             \item \textbf{Encoder:} \textbf{PositionalEncoding} と N個の\textbf{EncoderBlock}から構成[4]。
%             \item \textbf{Permute:} 埋め込み次元と時間軸を入れ替え(Batch, embed_dim, Time)[4]。
%             \item \textbf{AdaptiveAvgPool1d:} 時間軸でグローバル平均プーリング(1)[4]。
%             \item \textbf{Linear (Classifier):} 埋め込み次元をクラス数 (`n_classes`) に変換[4]。
%             \item \textbf{Sigmoid:} マルチラベル出力のため[4]。
%         \end{itemize}
%     \end{itemize}
% \end{frame}

\begin{frame}{EncoderBlockの構成} % 詳細な説明は難易度を上げる部分で、その後難易度を下げる [22]
    \begin{itemize}
        \item EncoderBlockは、以下を繰り返す構造です[4, 5]:
        \begin{itemize}
            \item \textbf{MultiHeadAttention (MHA):} 複数の注意機構で特徴を捉える[4]。
            \item \textbf{Add \& LayerNorm:} 残差接続と層正規化[4]。
            \item \textbf{FeedForward Network (FFN):} 全結合層[5]。
            \item \textbf{Add \& LayerNorm:} 再び残差接続と層正規化[5]。
        \end{itemize}
        \item このEncoderBlockが\textbf{N回}繰り返されることで、複雑な音響パターンを学習します[4]。
    \end{itemize}
\end{frame}

\begin{frame}{学習と評価の状況} % 現在の進捗を具体的に [2]
    \begin{itemize}
        \item \textbf{学習:}
        \begin{itemize}
            \item 作成したモデルを、事前に選定した9048データで訓練[6, 7]。
            \item \textbf{ハイパーパラメータ例:}[6, 7]
            \begin{itemize}
                \item `embed_dim=128` (埋め込み次元)
                \item `nhead=4` (Attentionのヘッド数)
                \item `nhid=256` (FFNの隠れ層)
                \item `nlayers=2` (エンコーダ層数)
                \item `epochs=20`, `lr=1e-3`, `val-split=0.1`
            \end{itemize}
        \end{itemize}
        \item \textbf{評価:}
        \begin{itemize}
            \item 評価データ修正済みだが未検証[7, 8]。
        \end{itemize}
    \end{itemize}
\end{frame}

% ==============================================================================
\section{研究目的の再考とサーベイ} % [8]
% ==============================================================================
\begin{frame}{研究意義の深掘り} % 発表者が「伝えるべき内容」を伝える [17]
    \begin{itemize}
        \item 研究目的の深掘りのため、調査を進めています[8]。
        \item \textbf{技術ありき}で話を進めていたため、\textbf{研究意義の説明が不足}している点が課題です[8]。
        \item 現在、関連論文を読み込み、研究意義の再考を行っています[8]。
        \begin{itemize}
            \item 院試用研究計画書の背景を再考する[8]。
            \item 卒論の研究意義再考のため調査する[8]。
            \item サーベイ論文を3本程度確認し、そこから重要論文を精読予定[8]。
        \end{itemize}
        \item 後で、再考した内容を要約し、より良い説明を作成します[8]。
    \end{itemize}
\end{frame}

% ==============================================================================
\section{異常イベントの音響データ収集} % [8]
% ==============================================================================
\begin{frame}{必要な異常音データと既存データセット調査} % 具体的な課題とアプローチ [22]
    \begin{itemize}
        \item 音特有の異常イベントデータ(例:呼吸、咳き込み、悲鳴など)の収集が必要です[8]。
        \item 現在、既存データセットの捜索を行っていますが、いくつか課題があります[8]。
        \item \textbf{調査済みのデータセット:}
        \begin{itemize}
            \item \textbf{Deeply Nonverbal Vocalization Dataset:} 作者への連絡が必要[8, 9]。
            \item \textbf{Respiratory Sound Database:} デジタル聴診器の音源が多く、行動認識には不向きな可能性[9]。AKG C417L Microphoneの音源のみ利用検討[9]。ライセンス情報なし[9]。
            \item \textbf{SAFE:} 転倒音データ[9]。
            \item \textbf{TAME Pain Dataset:} 痛みに関連する音声データ[9]。
            \item \textbf{Sound-Dr Dataset:} 内容が不明確[9]。
            \item \textbf{F2LCough:} 咳の分類データ[9]。
        \end{itemize}
        \item まずは咳や呼吸音を中心に捜索を進めます[9]。
    \end{itemize}
\end{frame}

\begin{frame}{自作データセット構築の懸念} % 懸念点を正直に提示する [43]
    \begin{itemize}
        \item \textbf{懸念:} \textbf{自作データセットの検証をどう行うか?}[10]
        \begin{itemize}
            \item 仮にデータセットを構築した場合、それで学習したモデルをどのように評価するか?[10]
            \item どうすれば\textbf{実用に耐えうるモデル}であることを示せるのか?[10]
            \item \textbf{評価方法}について検討する必要があります[10]。
        \end{itemize}
        \item \textbf{対応方針:}
        \begin{itemize}
            \item 現実的な想定のデータが見つかれば、それを優先的に使用[11]。
            \item あるいは、データセットに関する論文を参考に、現実的なラベルの組み合わせを考慮し、時間をかけてデータを構築すれば問題ないか[11]。
            \item \textbf{データセット構築段階が鍵}となると考えています[11]。
        \end{itemize}
    \end{itemize}
\end{frame}

% ==============================================================================
\section{今後のステップと課題} % [10]
% ==============================================================================
\begin{frame}{各種対応策の検討} % 具体的な行動計画 [2]
    \begin{itemize}
        \item 各種対応策を簡単に試行し、何をすべきか、どのくらい時間がかかるかを把握します[10]。
        \item \textbf{データ不均衡への対応:}
        \begin{itemize}
            \item 損失関数(例:\textbf{Focal loss}とその拡張)の変更を検討します[10]。
            \item シンプルな対応で済む可能性も考慮します[10]。
        \end{itemize}
        \item \textbf{半教師あり連合学習 (Federated Semi-Supervised Learning)} の導入[10]。
        \item \textbf{異常検知モデルの構築}[10]。
        \item \textbf{シミュレーション}方法の検討[10]。
    \end{itemize}
\end{frame}

\begin{frame}{今後のサーベイとデータ収集} % 今後の方向性 [1]
    \begin{itemize}
        \item \textbf{AALサーベイの継続:}
        \begin{itemize}
            \item 見守りという方向性に変更はありません[10]。
            \item 施設自体の方向性も維持し、AALがそういった研究分野であることを再確認します[10]。
            \item 関連論文を漁り、類似研究がないかを探しながら研究目的をさらに深掘りします[10]。
        \end{itemize}
        \item \textbf{データ収集の継続:}
        \begin{itemize}
            \item 各種データセットの分析と確認を続けます[10]。
            \item 新たなデータセットを探すため、音ならではの異常の兆候を多角的に考えます[10]。
        \end{itemize}
    \end{itemize}
\end{frame}

% ==============================================================================
\section{まとめ} % [1]
% ==============================================================================
\begin{frame}{本研究計画の要点}
    \begin{itemize}
        \item \textbf{目的:} スマートホームにおける\textbf{異常音検知システム}の実現可能性を追求し、在宅介護・医療支援に貢献する[1]。
        \item \textbf{進捗:} \textbf{マルチラベル音響検知モデルのPoC}を実装し、基本的な学習・評価環境を構築[2]。
        \item \textbf{課題:}
        \begin{itemize}
            \item \textbf{研究意義の深掘り}[8]:技術ありきから、社会貢献に焦点を当てた説明へ。
            \item \textbf{異常イベントデータの不足}[8]:信頼性の高いデータセットの確保と、自作データセットの評価方法確立。
            \item \textbf{データ不均衡・FL・異常検知モデル}など、具体的な技術的課題への対応[10]。
        \end{itemize}
        \item \textbf{今後:} これらの課題を解決し、ロードマップに沿って研究を具体化します[1, 10]。
    \end{itemize}
\end{frame}

% 質疑応答スライド [35, 64]
\begin{frame}{ご清聴ありがとうございました}
    \begin{center}
        \Large \textbf{ご清聴ありがとうございました!} \\
        \vspace{1cm}
        \Large \textbf{ご質問やご意見を承ります}
    \end{center}
    \vspace{0.5cm}
    \begin{itemize}
        \item 発表へのご質問、ご意見がございましたらお気軽にお声がけください[43]。
        \item スライドに関するご質問は、\textbf{各セクションの目次}をご参照ください。
    \end{itemize}
\end{frame}

\end{document}